\chapter{Introduction}

\section{The Problem}

In the past two decades, perhaps as a result of the culmination of the DevOps and Agile cultural movements, business IT tools and the information these generate have proliferated exponentially.  %reference needed %

Toolchains are now extensive, encompassing task and defect tracking systems (e.g. Jira, GitHub issues), Version Control Systems (e.g. Git, Mercurial), Continuous Integration \& Delivery (CI/CD) platforms (e.g. Jenkins, Travis CI), documentation (e.g. Confluence, Wiki Pages), deployment, configuration, orchestration and a plethora more.

% reference needed demonstrating the categorisations of toolchain components, either from state of devops report or Project to product.

Each tool presents its own idiosyncrasies, each providing different approaches to performing a given task on a subset of information or data. Often their capabilities overlap with each other. However, this duplication is not necessarily negative, as individual preferences on what is helpful vary on a personal or even team basis.

There are a wide number of papers illustrating the large amount of data that can be mined from developers' interactions with these toolchains - academically categorised as Mining Software Repositories - yet this data suffers the same fate as its source.

% appendix material needed showing the number of journals here...

This increase in information interaction entry-points, in pursuit of a more informed decision and improved experience, can result in a significant competition for the mental resources applied task at hand, often yielding varying returns at the expense of the exhaustion of a finite mental resource.

Much of the usefulness that can be sourced both from the tools themselves, in addition to the derivative sourced academic counterparts, can fall short of providing any real ecological use to the developer given a lack of immediate availability to the context in hand.

Software Developers focus around the production of software artefacts through the construction of source code of a given grammar and syntax. This was once a pen and paper endeavour, transitioning towards text-editors and most recently to rich text editors (RTE) and Integrated Development Environments (IDE); this evolution granted a greater context into the piece of code being edited by a developer, theoretically enabling more informed decisions and easier manipulation of given said source.

Framing this scope to encompass external context - in addition to the existing local focus - has produced extensions which integrate external information sources into the environment. Not withstanding this progress, the integration of these is as fragmented and inconsistent as the direct retrieval the sources which they interact with; pairing this with incomplete support matrix across environments and the inherent problems begin to emerge.

% ultimately tie this back to why are we using the IDE..

\section{The Vision}

To develop a common platform where context from any source can be \textbf{easily} integrated, aggregated and presented to the developer's working context when it is appropriate and, or needed to ultimately positively influence problem solving. 

\section{The Goals}

\newcommand{\printGoals}{
\label{Objectives}
\begin{enumerate}
	
	\label{objective:1}
	\item[\#1] Develop a working IDE extension for at least one IDE that is capable of presenting the aggregated integrated information based on a relevant context for a particular topic or motif. 
	
	\item[\#1a] Facilitate the expansion of the historically local context scope of the IDE in favour of a more global, organisational, scope.
	
	\label{objective:2}
	\item[\#2] Establish a common integration point to ingest mined information from a given source. Imperatively, this must be modular and easily extensible to allow for a variety of data to be mined.
	
	\label{objective:3}
	\item[\#3] Establish an approach where logic implemented for retrieval, aggregated and processing can be re-used across multiple different IDE.
	
	\label{objective:4}
	\item[\#4] Reduce overall time taken to find relevant information for a given problem context.
	
	\label{objective:5}
	\item[\#5] Contrast the suitability of different information delivery mechanisms; concretely exploring, requesting - pulling - versus being fed - pushing - information, when it is somehow determined to be relevant. 
	
\end{enumerate}
}

\printGoals

\subsection{The Contribution}

In satisfaction of the requirements specified for this dissertation, the key aim is to deliver a designed, implemented and evaluated software solution that addresses the above goals with the intention of enabling the stated vision. We aim to deliver an IDE extension that is capable of presenting information that we have aggregated from multiple different sources based on a given topic or motif. We aim for this aggregation to be implemented as a separate distributed application which is modularly extensible and easily deployable.