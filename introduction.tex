\chapter{Introduction}

\section{The Problem}

\normalfont There are a wide number of papers illustrating the large amount of data that can be mined from developers' interactions with their toolchain. This tool-chain often includes task and bug tracking systems (e.g. Jira, GitHub issues), Version Control Systems (e.g. Git, Mercurial), Continuous Integration \& Delivery (CI/CD) platforms (e.g. Jenkins, Travis CI) and general declarative or/and imperative based infrastructure.

This activity of exploring this data is known as Mining Software Repositories.

Much of this information falls short of its useful potential as it is not made readily available to the developer and thus further falls short of proving any real ecological validity.

\section{Aims and Objectives}

The aim of this paper is to develop a common platform through which mined content can be presented, aggregated by topics or motif, to the developer through an Integrated Development Environment's (IDE) context mechanisms. There is a key interest to expand from the historic focus on local context scope towards a global organisational scope.

A large part of this work will operate on the assumptions that developers want or/and will use the information presented to them. Further, we pose questions as to whether it is best to take a push model (where information is presented based on IDE actions) or a pull model (resembling a more traditional query approach, where developers request information). Discussing and evaluating the appropriateness of each will be of substantial importance. 

This defines a minimum viable product demonstrating an interaction mechanism in an IDE whereby a developer can query a variety of mined data based on a given agreed topic; this interaction mechanism should be supplemented by a modular, extensible data feed implementation. Proving the technical viability is a key aim of this paper. 