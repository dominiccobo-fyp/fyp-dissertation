\chapter{Introduction}

\section{The Problem}

\normalfont There are a wide number of papers illustrating the large amount of data that can be mined from developers' interactions with their toolchain. This tool-chain often includes task and bug tracking systems (e.g. Jira, GitHub issues), Version Control Systems (e.g. Git, Mercurial), Continuous Integration \& Delivery (CI/CD) platforms (e.g. Jenkins, Travis CI) and general declarative or/and imperative based infrastructure.

This activity of exploring this data is known as Mining Software Repositories.

Much of this information falls short of its useful potential as it is not made readily available to the developer and thus further falls short of proving any real ecological validity and ultimately providing any benefit to the world.

\section{Setting the Vision}

To develop a common platform where context from any source can be EASILY integrated, aggregated and presented to the developer's working context when it is appropriate and, or needed to ultimately assist problem solving. 

\section{Establishing the Goals to Achieve the Vision}

\begin{enumerate}
	\item[\#1] Develop a working IDE extension for at least one IDE that is capable of presenting the aggregated integrated information based on a relevant context for a particular topic or motif.
	
	\item[\#1a] Expand the historically local context scope of the IDE in favour of a more global, organisational, scope.
	
	\item[\#2] Establish a common integration point to ingest mined information from a given source. Imperatively, this must be modular and easily extensible to allow for a variety of data to be mined.
	
	\item[\#3] Establish an approach where logic implemented for retrieval, aggregated and processing can be re-used across multiple different IDE.
	
	\item[\#4] Reduce overall time taken to find relevant information for the present problem context.
	
	\item[\#5] Contrast the suitability of requesting - pulling - versus being fed - pushing - information, when it is somehow determined to be relevant. 
	
\end{enumerate}