\chapter{Introduction}

\section{The Problem}

In the past two decades, perhaps as a result of the culmination of the DevOps and Agile cultural movements, business IT tools and the information these generate have proliferated exponentially.  %reference needed %

Toolchains are now extensive, encompassing task and defect tracking systems (e.g. Jira, GitHub issues), Version Control Systems (e.g. Git, Mercurial), Continuous Integration \& Delivery (CI/CD) platforms (e.g. Jenkins, Travis CI), documentation (e.g. Confluence, Wiki Pages), deployment, configuration, orchestration and a plethora more.

% reference needed demonstrating the categorisations of toolchain components, either from state of devops report or Project to product.

Each tool presents its own idiosyncrasies, each providing different approaches to performing a given task on a subset of information or data. Often their capabilities overlap with each other.

There are a wide number of papers illustrating the large amount of data that can be mined from developers' interactions with these toolchains. This activity is academically categorised as Mining Software Repositories.

% appendix material needed showing the number of journals here...

This increase in interaction entry-points in pursuit of a more informed decision can result in a significant detraction from the task at hand, often yielding varying degrees of success. 

Much of the usefulness that can be sourced both from the tools themselves, in addition to the derivative sourced academic counterparts, can fall short of providing any real ecological use to the developer given a lack of immediate availability to the context in hand.

\section{The Vision}

To develop a common platform where context from any source can be EASILY integrated, aggregated and presented to the developer's working context when it is appropriate and, or needed to ultimately positively influence problem solving. 

\section{The Goals}

\begin{enumerate}
	\item[\#1] Develop a working IDE extension for at least one IDE that is capable of presenting the aggregated integrated information based on a relevant context for a particular topic or motif. 
	
	\item[\#1a] Facilitate the expansion of the historically local context scope of the IDE in favour of a more global, organisational, scope.
	
	\item[\#2] Establish a common integration point to ingest mined information from a given source. Imperatively, this must be modular and easily extensible to allow for a variety of data to be mined.
	
	\item[\#3] Establish an approach where logic implemented for retrieval, aggregated and processing can be re-used across multiple different IDE.
	
	\item[\#4] Reduce overall time taken to find relevant information for the present problem context.
	
	\item[\#5] Contrast the suitability of different information delivery mechanisms; concretely exploring, requesting - pulling - versus being fed - pushing - information, when it is somehow determined to be relevant. 
	
\end{enumerate}