\chapter*{Abstract}

The information and tools interacted with by developers, for daily problem solving, has exponentially proliferated in recent years. 

This increase is not without consequence. Navigating, filtering and using the required information is cognitively tasking and time consuming. The quest for more informed choices, in knowledge of the mere existence of useful information, can often distract from the original task at hand. 

Mechanisms for enrichment of developer's context exist, but require separate implementation efforts across different environments, introducing apparent usage barriers.

In this project we seek to create a platform that facilitates the integration of a variety of sources with an easy, replicable implementation across environments, producing tailored queries and yielding responses for a given problem context.  

\chapter*{Acknowledgements}

\chapter*{Outline}

The contents of this work are best read in chronological order, front to end. The content is synthesised in a manner where the each section builds upon knowledge divulged in previous sections, building from abstract high-level explanations, to concrete, well-founded statements. Ergo, what may seem like an endless barrage of questions, is a rhetoric guidance to our thought process and is juxtaposed with conclusions in each stage.

In Chapter 1, the introduction, starts by exploring the problem, its context and providing a clear definition. This is then followed with an articulation of the vision set out to remedy the problem, finalised with the goals and our contribution.

In Chapter 2, that which is written in the Introduction, is explored through research, providing an in-depth exploration of the problem, from abstract to concrete. This section also clarifies the research philosophy adopted during this work.

Chapter 3 briefly provides an explanation of the non-linear approach which has lead to the construction of this text and the artefacts which accompany it.

Chapter 4, constitutes the main body of the dissertation, designing, implementing and evaluating iteratively.

Chapter 5, summarises the evaluations performed throughout Chapter 4, supporting with further overarching evaluation. Finally a roadmap which the developed product would have followed providing more time, is explored.