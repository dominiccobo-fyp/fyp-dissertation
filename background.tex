\chapter{Background}
\section{How do we know the problem exists?}

The particular topic of Mining Software Repositories has its own conference \parencite{Storey:2019:3341883} which means that by assessing the assumptions made by these, a trend is apparent that the papers will go as far as creating tools, uncovering information or general datasets, but fall short of making it immediately useful in routine decision-making in Developer's problem solving.

\section{How do we know aggregated sources [e.g. in an IDE] helps productivity?}

An Integrated Development Environment by definition aims to integrate a wide variety of tools used to develop software into a single cohesive scope. 

A by no means comprehensive exploration of literature from several sources suggests that IDEs and centralised information assist developers positively. 

Literature from the popular Software Engineering teaching resource Software Engineering \parencite{Sommerville:2010:SE:1841764} page 37, states that "IDEs provide a common platform to communicate"; in page 197, he further elaborates on this suggesting "IDEs are a set of software tools supporting software development".

Kersten's Project to Product \parencite*{kersten2018projecproduct} identifies the effect on the delivery of business value with the presence of a complex delivery chain of information, tools and processes.

Researching psychology studies on the effect of information fragmentation on a person's ability to complete a task provides ample insight. 

The research presented by Klemola and Riling \parencite*{klemola2002modeling} suggests that searching and using different tools forms part of the cognitive load that a developer must handle.

Sweller \parencite*{sweller1998cognitive} identifies that problem solving requires a significant amount of cognitive load. Miyake and Shah \parencite*{miyake1999models} explore the notion of limited working memory - which by extension establishes that there will be competition for cognitive resources. This shows that for developers, it is hypothetically detrimental to waste precious resources for problem solving on searching fragmented information.


\section{How do we know the implementation is possible with current IDEs?}

Looking at the push-based approach discussed in the Introduction, we may consider the use of the Language Server Protocol (LSP) \parencite{lspGitHubSiteMSFT}.

LSP provides a slim-lined model for solving the M x N language compatibility issue for IDEs. This means that contextual model can be developed once for a model and shared to all IDEs. The contextual model can be developed for any grammar once and shared for many IDEs. LSP is supported by most major IDEs.

A consideration of the pull-based approach, we may consider extending the ideas developed by the Language Server Protocol to extend to the action and command APIs provided by most major IDEs (table \ref{table:1}).

\begin{table}
\centering
\begin{tabular}{ | c | c | c |}
	\hline
		IDE & Actions/Commands Support & LSP API Available?\\
	\hline
	\hline
		JetBrains IDEs & Yes & Yes, via LSP4J\\ 
		Visual Studio Code & Yes & Yes, via vscode-languageclient \\
		Atom & Yes & Yes, via atom-language client.\\
		Eclipse & Yes & Yes, via LSP4J\\
	\hline
\end{tabular}
\caption{Action and LSP support across IDEs}
\label{table:1}
\end{table}


\section{Are there any existing efforts in this field?}

Papers from Mining Software Repositories illustrate the growing interest in extracting information from the toolchain, but present little exploration of delivering this information to aid periodic decision making. 

%TODO: insert table / graph showing growth in interest supporting above claimm.

The IEEE Transactions on Software Engineering Journal Series, presents a large amount of empirical research on the challenges faced in the world of Software Engineering. 

%TODO: how do we reference a series of journals?

The state-of-the-art-DevOps practices from annual conferences, such as All Day DevOps and Puppet's State of DevOps report, focus on the cultural change where all stakeholders are empowered to share more information to reduce feedback time-cycles when solving problems.

%TODO: how do we reference a public report series such as the above or non academic conferences.