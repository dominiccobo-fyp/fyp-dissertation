\chapter{Background}

\section{On the Problem}

\subsection{How do we know the problem exists?}

Coming to terms with the ontology and epistemology of the problem do we realise, in a problem-driven approach, the importance of not limiting the streams of how we obtain information, nor constraining what is constituted as valid to that which is only of academic origin.

The intrinsic nature of the selection of tools is indicative of a personal preference on what we as humans consider individually useful. However, simultaneously the use of tools is theoretically supportive to the production and, or modification of a given software artefact to ultimately deliver business value - a measurable, quantifiable notion.

Similarly, the disconnection existing between academics - favouring empirical approaches - and industrial practitioners - valuing a mixture of sources based on experience - can serve as a source of how problems such as ours are investigated.

This suggests that what we constitute as the existence and exploration of reality and knowledge, in this case, is not a binary choice between ontological subjectivism and objectivism but rather a mixture of the two. Drawing from the strengths of both are we best enabled to contrast the different aspects of the problem. Understanding the objective nature of Realism, Empiricism and Positivism alongside the solely subjective nature of Interpretivism do we ultimately favour the stance of Pragmatism.

\subsection{Is the proliferation documented?}

% TODO: expand on these points...

Recent market research suggests that DevOps related tools are expected to grow 24.7\%, from \$2.90bn to \$10.31bn between 2017 and 2023.

Market growth rate of tools related to DevOpsy things is growing at a far greater pace than other tools. https://devops.com/downloads/the-state-of-devops-tools-2019/

\citeauthor{sodrPuppet2017} in the Puppet State of DevOps industrial community reports (\citedate{sodrPuppet2017})(\citedate{sodrPuppet2014}) reveal how individual teams being granted autonomy to choose tooling is highly important to the effectiveness of good tooling; an individual choice across many teams through the expanse of a large organisation may lead to an increased demand for a choice of tools.

Through unfortunate misunderstanding of what the cultural movements of DevOps and Agile mean, do some companies opt to spend on tools over addressing core problems, again resulting in an increase of demand.

% mention economics laws of how demand for something may affect supply? 
\cite{sodrAccelerate2018} identifies that often in pursuit of the benefits of DevOps there is a disconcerting focus on investing in tools over addressing the core problems. This interest for off-the-shelf, buzzword compliant solutions is also identified in the earlier 2015 report \parencite{sodrPuppet2015}.

With digital disruption leaving not even the giants unscathed \parencite{weForum2016}, the demand for products to aid digital business is in no shortage with companies believing that there mere adoption will sustain their survival.

A very thorough evaluation carried out by \cite{kersten2018cambrian} directly explores the proliferation of tools related to DevOps and the resulting effects of diversity. He further identifies that there are negative and positive types of diversification.

Supporting the accentuation of this growth, as a company working on integrating these tools, \cite{xebiaLabs2019DevOpsReflections} from Xebia Labs, identify the extent of the proliferation on existing taxonomical approaches for recommending adoption.

\subsection{Does the adoption of these tools aid problem solving?}

%TODO;

Machines excel at reproducing repetitive logical, computational steps defined by humans. Reducing our involvement in repetitive tasks liberates our to perform tasks that require the creative cognitive abilities that until now only we posses.  

This jump, now occurring in the tertiary sector is not unique and has long since been experienced in the primary and secondary sectors. 

Car manufacturing is an example of this very phenomena where through the propagation of the Lean principles installed initially by the TPS, were repetitive tasks allocated to machines for humans to focus on the tasks they do better.% TODO: find references.

However, it is important to note that the adoption of tools alone is not a silver bullet to all woes and worries in an organisation. 

A 2014 report by \citeauthor{azhari2014digital} identifies the multidimensionality of digitisation of work places as more than just tool adoption, instead consisting of being a more intricate mixture of technology, governance, people, culture, operations, products, leadership and strategy.

In Chapter 30 of Code Complete \parencite{codeComplete}, a widely accepted guideline for Software Development, the author identifies studies by which show that the usage of certain tools can help boost productivity by 50 percent or more (\cite{jones2000software}; \citeauthor{boehm2000software} \citeyear{boehm2000software}).

%TODO: reference here on usage of search engines

Albert Einstein once said: "Never memorise something that you can look up"; We developers use search engines repeatedly for reference purposes to refresh our minds on smaller intricacies of the syntax, language or problem context that we are working with. It only takes simple observation of any developer, experienced or inexperienced, to see the frequency of asserting their knowledge with the aid of external resources.

Task management systems.

- assist in documenting and coordinating tasks, defects and changes in a complex dynamic business environment.
- work in these however can be buried, or "tidied away" through infinite taxonomical structure leading to an organised mess.

Automated build systems 

- these provide declarative, reproducible steps that enable a greater transparency into the assembly, testing and distribution of software artefacts.

Messaging systems.

Version control systems.

- these enable developers to be 

Documentation systems.

- these enable the versioned, synchronised, sharing of documentation related to a product for reference purposes

Monitoring / analytic tools.

\subsection{On overlapping and duplicated capabilities}

Much of the confusion around IT tools occur as a result of the duplication of functionality. This is not unique to computing. In the culinary world, there are hundreds of different types of knives, each ultimately achieves the same goal to process food but each user has a different preference. For novice cooks the confusion this degree of choice leads to is inevitable; the experienced cook, in contrast, has the knowledge to make an informed choice on the appropriate variety for the problem at hand. It is the challenge of any craftsperson of any domain to choose the right tool for the job.

Freedom to choose the suited tool by an experienced craftsperson, ultimately bestows an investment that may have been absent had this election been a forced, curated suggestion.

% TODO; 

Research specific to the IT industry supports this analogy, with annual surveys such as the State of DevOps Report (\citedate{sodrPuppet2014}) (\citedate{sodrPuppet2017})  highlighting the importance of tailored, personal choice. 

The issue we experience is not to with controlling the natural consequences of growth that have led to the proliferation of tools, but rather than inappropriate selection, difficult access to or disconnection from their purpose results in issues. 

%Disconnection can cause misalignment in value streams - mik kersten

\subsection{How has this proliferation been received by researchers?}

The large adoption, but not necessarily proliferation, of tools has not gone unnoticed in academia. The usage of tools generates an innumerable sum of data of immense use in supporting research.

Academic journals in Mining Software Repositories - e.g. \cite{Storey:2019:3341883} - have focused on exploiting this. Many of the papers in this field produce new tools, uncover new information or create new datasets, but generally fall short of make it readily available to routine decision making for the developers.


% TODO: 
http://portal.core.edu.au/conf-ranks/711/

For industry practitioners and researchers, these tools have provided the base for measuring against particular benchmarks and, or within certain frameworks as a way of supporting improvement and advances. 

% TODO: examples of benchmarks provided by tools.

Recent referential material included on this includes Accelerate, Project to Product, The DevOps Handbook etc

\subsection{How does fragmentation affect problem solving?}

%Two sides of this: we have the effect on product delivery

Kersten's Project to Product \parencite*{kersten2018projecproduct} identifies the effect on the delivery of business value with the presence of a complex delivery chain of information, tools and processes.

% and later on the developer's ability

Researching psychology studies on the effect of information fragmentation on a person's ability to complete a task provides ample insight. 

The research presented by Klemola and Riling \parencite*{klemola2002modeling} suggests that searching and using different tools forms part of the cognitive load that a developer must handle.

Sweller \parencite*{sweller1998cognitive} identifies that problem solving requires a significant amount of cognitive load. Miyake and Shah \parencite*{miyake1999models} explore the notion of limited working memory - which by extension establishes that there will be competition for cognitive resources. This shows that for developers, it is hypothetically detrimental to waste precious resources for problem solving on searching fragmented information.

\subsection{On the evolution of the craftsmanship of software}

Fortran punch cards handed to developer

Text editors

Integrated development environments 


\section{On the Goals}

\subsection{Context in a Developer's Environments}

An Integrated Development Environment by definition aims to integrate a wide variety of tools used to develop software into a single cohesive interaction pane. 

A by no means comprehensive exploration of literature from several sources suggests that IDEs and centralised information assist developers positively. 

Literature from the popular Software Engineering teaching resource Software Engineering \parencite{Sommerville:2010:SE:1841764} page 37, states that "IDEs provide a common platform to communicate"; in page 197, he further elaborates on this suggesting "IDEs are a set of software tools supporting software development".

The context provided within an IDE mainly focuses on the edition, manipulation and referential assistance for code; this is despite the suggestion that IDEs provide a platform to communicate.

\citeauthor{kersten2006using} (\citedate{kersten2006using}) conclude in a study on enriching a developer's context using task-based context through various mechanisms, positively influences developer's productivity.

\subsection{Comparing Extension in IDEs}

Text Editors and Integrated Development Environments can only cater for a finite set of user requirements before a user's needs deviate from the beaten path. Rich text editors take bare-bones philosophy, providing little more than essential text editing abilities out of the box; in contrast, IDEs provide a much more tailored out-of-the-box experience, generally tailored to particular language. 

Extension of existing functionality naturally requires a connecting programmatic interface from which to start from. Naturally, the functionality and approach to these extensions varies vastly from IDE to IDE, with standardisation seeming unlikely. 

Looking at the most popular development environments, according to the annual StackOverflow Developer report \parencite{stackOverflowDevReport2019}, we dive into a comparison of each.

Examining previous reports in this series: 

https://insights.stackoverflow.com/survey/2018

https://insights.stackoverflow.com/survey/2017

https://insights.stackoverflow.com/survey/2016

We can see a change in popularity.

% TODO: insert these as footnote references ...

% https://marketplace.visualstudio.com/search?target=VSCode&category=All%20categories&sortBy=Installs

% https://plugins.jetbrains.com/

% https://marketplace.eclipse.org/

% https://www.jetbrains.com/company/

% https://archive.eclipse.org/eclipse/downloads/drops/R-1.0-200111070001/

% https://github.com/microsoft/vscode/commits/master

% https://www.jetbrains.org/intellij/sdk/docs/basics/getting_started/creating_plugin_project.html

% https://code.visualstudio.com/api

% https://www.eclipse.org/pde/


\begin{table}[h!]
\centering
\begin{tabular}{|l|l|l|c|l|}
	\hline 
	\rule[-1ex]{0pt}{2.5ex} Name & Language & Platform & No. of Extensions & Released \\ 
	\hline 
	\hline 
	\rule[-1ex]{0pt}{2.5ex} Visual Studio Code & Javascript & Node & ~16400 & Q4 2015\\ 
	\hline 
	\rule[-1ex]{0pt}{2.5ex} JetBrains IntelliJ & Java & OSGi  & ~4238 & Q1 2001 \\ 
	\hline 
	\rule[-1ex]{0pt}{2.5ex} Eclipse IDE & Java & OSGi & ~1713 & Q4 2001 \\ 
	\hline 
\end{tabular} 
\caption{Comparison of IDE and their extensions}
\end{table}

This difference in language and integration approaches inherently means that functionality implemented in one IDE cannot be reused across another. Interestingly, we may note that even despite Eclipse IDE and JetBrains IntelliJ using the same language, even together they do not reach half the extensions available for Visual Studio Code.

\subsection{Are there any similar issues?}

The logic compatibility matrix issue is not an unexplored issue in the domain of IDEs and RTEs. In the not too distant past, each environment had a programming language which it primarily focused on addressing, other support was either lacking or implemented to a less full degree. 

This is known as the M x N language compatibility issue. For each environment, M, there exists the requirement to implement support for N given languages. Inherently, this results in the reinvention and, or reimplementation of the wheel for an identical requirement set bar the functional presentation and interaction.

Building upon existing understanding of the benefits obtained by decoupling presentation and logic, resorting to thin clients, does the Language Server Protocol \parencite{lspGitHubSiteMSFT} appear in recent years.

LSP provides a slim-lined model for solving the M x N language compatibility issue for development environments. The contextual model can be developed once for a given syntax and grammar and reused across all, providing these implement a very simple specification to handle interfacing with the user.

A consideration of the pull-based approach, we may consider extending the ideas developed by the Language Server Protocol to extend to the action and command APIs provided by most major IDEs (table \ref{table:1}).

\begin{table}[h!]
	\centering
	\begin{tabular}{ | c | c | c |}
		\hline
		IDE & Actions/Commands Support & LSP API Available?\\
		\hline
		\hline
		JetBrains IDEs & Yes & Yes, via LSP4J\\ 
		Visual Studio Code & Yes & Yes, via vscode-languageclient \\
		Atom & Yes & Yes, via atom-language client.\\
		Eclipse & Yes & Yes, via LSP4J\\
		\hline
	\end{tabular}
	\caption{Action and LSP support across IDEs}
	\label{table:1}
\end{table}

\subsection{Some interesting changes}

IDEs now available as web pages thanks to advances in web technology.

e.g. eclipse che, cloud9, vscode online etc

this, highlights that code editors are trending towards merely being presentation layers in a wider distributed architecture, catching up to technological advances that have been around for decades.

\subsection{On the usability of information}

Findability is the UX principle attributed to the ease of discovering known data, information using a search or browse approach.

Presenting information from multiple sources carries certain critical choices, such as whether to apply taxonomy or flatten . 

https://www.nngroup.com/articles/flat-vs-deep-hierarchy/