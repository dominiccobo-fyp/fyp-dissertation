\chapter{Background}

\section{On the Problem}

\subsection{How do we know the problem exists?}

Coming to terms with the ontology and epistemology of the problem do we realise, in a problem-driven approach, the importance of not limiting the streams of how we obtain information, nor constraining what is constituted as valid to that which is only of academic origin.

The intrinsic nature of the selection of tools is indicative of a personal preference on what we as humans consider individually useful. However, simultaneously the use of tools is theoretically supportive to the production and, or modification of a given software artefact to ultimately deliver business value - a measurable, quantifiable notion.

Similarly, the disconnection existing between academics - favouring empirical approaches - and industrial practitioners - valuing a mixture of sources based on experience - can serve as a source of how problems such as ours are investigated.

This suggests that what we constitute as the existence and exploration of reality and knowledge, in this case, is not a binary choice between ontological subjectivism and objectivism but rather a mixture of the two. Drawing from the strengths of both are we best enabled to contrast the different aspects of the problem. Understanding the objective nature of Realism, Empiricism and Positivism alongside the solely subjective nature of Interpretivism do we ultimately favour the stance of Pragmatism.

\subsection{Is the proliferation documented?}

% TODO: expand on these points...

Recent market research suggests that DevOps related tools are expected to grow 24.7\%, from \$2.90bn to \$10.31bn between 2017 and 2023.

Market growth rate of tools related to DevOpsy things is growing at a far greater pace than other tools. https://devops.com/downloads/the-state-of-devops-tools-2019/

\citeauthor{sodrPuppet2017} in the Puppet State of DevOps industrial community reports (\citedate{sodrPuppet2017})(\citedate{sodrPuppet2014}) reveal how individual teams being granted autonomy to choose tooling is highly important to the effectiveness of good tooling; an individual choice across many teams through the expanse of a large organisation may lead to an increased demand for a choice of tools.

Through unfortunate misunderstanding of what the cultural movements of DevOps and Agile mean, do some companies opt to spend on tools over addressing core problems, again resulting in an increase of demand.

% mention economics laws of how demand for something may affect supply? 
\cite{sodrAccelerate2018} identifies that often in pursuit of the benefits of DevOps there is a disconcerting focus on investing in tools over addressing the core problems. This interest for off-the-shelf, buzzword compliant solutions is also identified in the earlier 2015 report \parencite{sodrPuppet2015}.

With digital disruption leaving not even the giants unscathed \parencite{weForum2016}, the demand for products to aid digital business is in no shortage with companies believing that there mere adoption will sustain their survival.

A very thorough evaluation carried out by \cite{kersten2018cambrian} directly explores the proliferation of tools related to DevOps and the resulting effects of diversity. He further identifies that there are negative and positive types of diversification.

Supporting the accentuation of this growth, as a company working on integrating these tools, \cite{xebiaLabs2019DevOpsReflections} from Xebia Labs, identify the extent of the proliferation on existing taxonomical approaches for recommending adoption.

\subsection{Does the adoption of these tools aid problem solving?}

%TODO;

% M. Kersten and G.C. Murphy, “Using Task Context to Improve Programmer Productivity,” Proc. 14th ACM SIGSOFT Int’l Symp. Foundations of Software Eng. (SIGSOFT/FSE 06)

Machines excel at reproducing repetitive logical, computational steps defined by humans. Reducing our involvement in repetitive tasks liberates our to perform tasks that require the creative cognitive abilities that until now only we posses.  

This jump, now occurring in the tertiary sector is not unique and has long since been experienced in the primary and secondary sectors. 

% TODO: find references.

P. Azhari, N. Faraby, A. Rossmann, B. Steimel and K. S. Wichmann, “Digital transformation report,” neuland GmbH \& Co. KG., Köln, 2014 (pp 153) identifies the multidimensionality of digitisation of work places as more than just tool adoption, instead consisting of being a more intricate mixture of technology, governance, people, culture, operations, products, leadership and strategy.

\subsection{On overlapping and duplicated capabilities}

% TODO; 

2017 SoDR https://media.webteam.puppet.com/uploads/2019/11/2017-state-of-devops-report-puppet-dora.pdf reveals how individual teams being allowed to choose tooling is highly important... 

Disconnection can cause misalignment in value streams - mik kersten

\subsection{How does fragmentation affect problem solving?}

%TODO;

\subsection{How has this proliferation been received by researchers?}

The large adoption, but not necessarily proliferation, of tools has not gone unnoticed in academia. The usage of tools generates an innumerable sum of data of immense use in supporting research.

Academic journals in Mining Software Repositories - e.g. \cite{Storey:2019:3341883} - have focused on exploiting this. Many of the papers in this field produce new tools, uncover new information or create new datasets, but generally fall short of make it readily available to routine decision making for the developers.

For industry practitioners and researchers, these tools have provided the base for measuring against particular benchmarks and, or within certain frameworks as a way of supporting improvement and advances.

% TODO: quantification needed

\subsection{How do we know aggregated sources [e.g. in an IDE] helps productivity?}

An Integrated Development Environment by definition aims to integrate a wide variety of tools used to develop software into a single cohesive scope. 

A by no means comprehensive exploration of literature from several sources suggests that IDEs and centralised information assist developers positively. 

Literature from the popular Software Engineering teaching resource Software Engineering \parencite{Sommerville:2010:SE:1841764} page 37, states that "IDEs provide a common platform to communicate"; in page 197, he further elaborates on this suggesting "IDEs are a set of software tools supporting software development".

Kersten's Project to Product \parencite*{kersten2018projecproduct} identifies the effect on the delivery of business value with the presence of a complex delivery chain of information, tools and processes.

Researching psychology studies on the effect of information fragmentation on a person's ability to complete a task provides ample insight. 

The research presented by Klemola and Riling \parencite*{klemola2002modeling} suggests that searching and using different tools forms part of the cognitive load that a developer must handle.

Sweller \parencite*{sweller1998cognitive} identifies that problem solving requires a significant amount of cognitive load. Miyake and Shah \parencite*{miyake1999models} explore the notion of limited working memory - which by extension establishes that there will be competition for cognitive resources. This shows that for developers, it is hypothetically detrimental to waste precious resources for problem solving on searching fragmented information.


\section{On the Goals}

\subsection{Extensions with IDEs}

% TODO: document the difference in implementation support between IDEs and Rich Text Editors...

\subsection{Are there any similar issues?}

The logic compatibility matrix issue is not an unexplored issue in the domain of IDEs and RTEs. In the not too distant past, each environment had a programming language which it primarily focused on addressing, other support was either lacking or implemented to a less full degree. 

This is known as the M x N language compatibility issue. For each environment, M, there exists the requirement to implement support for N given languages. Inherently, this results in the reinvention and, or reimplementation of the wheel for an identical requirement set bar the functional presentation and interaction.

Building upon existing understanding of the benefits obtained by decoupling presentation and logic, resorting to thin clients, does the Language Server Protocol \parencite{lspGitHubSiteMSFT} appear in recent years.

LSP provides a slim-lined model for solving the M x N language compatibility issue for development environments. The contextual model can be developed once for a given syntax and grammar and reused across all, providing these implement a very simple specification to handle interfacing with the user.

A consideration of the pull-based approach, we may consider extending the ideas developed by the Language Server Protocol to extend to the action and command APIs provided by most major IDEs (table \ref{table:1}).

\begin{table}[h!]
	\centering
	\begin{tabular}{ | c | c | c |}
		\hline
		IDE & Actions/Commands Support & LSP API Available?\\
		\hline
		\hline
		JetBrains IDEs & Yes & Yes, via LSP4J\\ 
		Visual Studio Code & Yes & Yes, via vscode-languageclient \\
		Atom & Yes & Yes, via atom-language client.\\
		Eclipse & Yes & Yes, via LSP4J\\
		\hline
	\end{tabular}
	\caption{Action and LSP support across IDEs}
	\label{table:1}
\end{table}
