\section{Sprint 2}

\subsection{Sprint Planning}

\subsubsection{Impact Map}

\subsubsection{Selected User Stories}

\begin{enumerate}
	\item As a library / API consumer, to help my consumption, when I enter a command to query for examples, show me how others use it.
	
	\item As a library / API developer, to guide my development and increase awareness, when I enter a command to query for usages, show me how others use it.
	
\end{enumerate}

\subsubsection{Sprint Goal}

Have greater context of API usage implemented through a pull model using mock data.

\subsection{Sprint Implementation}

Moving away from the direct language server specification.

We're no longer interested in the direct support it provides.

We can however learn from its implementation, adapting / reusing the pattern it leverages to address the m x n IDE-language problem.

All previously examined IDEs provide some platform to . (see tab 2.1 background).

Query results are naturally no longer expected to displayed in the hover pop up within the IDE editor windows, so we must take another approach. Instead we consider the ability to render queries as documents; we'll take little care as to the formatting as we focus on the MVP.

%TODO;table of which IDEs support rendering some form of webview.

\subsection{Sprint Review}

More natural feeling, suited more to browsing and discoverability as opposed to being ``fed'' the information.

Proves the feasibility and suitable direction for the project.

\subsection{Sprint Retrospective}

\subsubsection{What went well?}

\subsubsection{What didn't go so well?}

\subsubsection{What do we want to improve?}