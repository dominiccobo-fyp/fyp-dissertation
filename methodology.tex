\chapter{Methodology}

There is an inherent complexity in exploring and solving a problem in an unknown domain and delivering a solution within a defined time frame. We are projecting a product to provide a solution to a problem we do not fully understand. 

Literature, suggests that a ``big upfront" design is unsuitable for tackling projects in an unknown domain. Taking an agile approach and focusing on delivering iteratively and incrementally to build out a product to solve the problem, allows this uncertain complexity to be tackled in a more manageable fashion. 

By focusing on testing hypothesis against smaller changes to a product, we can draw a more isolated conclusive evaluation.

Whilst it is not necessary to use a particular methodology to be agile in approach, doing so, provides a sound base for building upon. One must remember agility relies on people and interactions with a preference over tools and processes. 

There are many ``agile-compliant methodologies" in existence: These include, but are not limited to: Extreme Programming, Scrum, Kanban, RAD and DSDM.

% TODO: is a comparison of the approaches suitable?

% TODO: dirrect qute %%
The Scrum Process Framework provides a strong foundation for developing, delivering and sustaining complex products. 

% TODO: applying scrum 

\section{Applying Scrum}

\subsection{Mapping The Roles}

\begin{description}
	\item[Scrum Master] The Scrum Master's role in nature is pastoral making it fitting to the supervisor of this paper.
	
	\item[Product Owner] Holding the primary roles of maximising the value of the product and managing the backlog, this role is suitably allocated to the paper author.
	
	\item[Development Team] Responsible for the implementation of any backlog item. The sole developer for the project being the author of this paper.
\end{description}

\subsection{Mapping The Events}

\begin{description}
	\item[The Sprint] The time-boxing of development towards a potentially releasable increment and a definition of done, is well suited to match the period between supervisor meetings. This qualifies traditionally as implementation of a solution.
	
	\item[Sprint Planning] Sprint planning, an activity where work to be done is allocated and designed, is suited to supervisor meetings. 
	
	\item[Daily Scrum] The Daily scrum focuses on inspecting progress towards Sprint Goals, allowing team members to update and communicate progress, intentions and impediments to self organise and increase the chance of meeting the Sprint Goal. This event is very team and communication centric, something that is very difficult to implement in a single-person project that this context is.
	
	\item[Sprint Review] Focusing on sharing the increment with stakeholders and evaluating its suitability and introducing further work to refine if needed. This forms part of the constant evaluation towards solving the problem stated. 
	
	\item[Sprint Retrospective] Focusing on evaluating the process rather than the potentially shippable increment, this allows a refinement towards the approach itself.
	
\end{description}

% TODO: do we really need to map the artefacts, they are self explanatory.

%\subsection{Mapping The Artefacts}
%
%\begin{description}
%	\item[Product Backlog] description
%	\item[Sprint Backlog] description
%	\item[Increment] description
%\end{description}

\subsection{Defining Done}

An increment is considered to be done when it is checked in, packaged and deployed, ready for demonstration to the end user. 