\chapter{Conclusions}

\section{Concluding the iterative process}

\section{Concluding on the overall solution}

% TODO:

% This is where you draw your final conclusions. You have presented your findings or data, now summarise how you have met each objective, and draw a conclusion as to whether you have met your overall aim.  You should provide some justification for this.  There are three possibilities here:

%1. You have completely met your aim, and solved your problem (unlikely)
%2.  Your results show that your solution does not solve the problem at all (unlikely)
%3. You conclude that your solution addresses your problem to some extent, but that there are weaknesses in the approach in other regards (most likely)

%In each case, you will have produced a valid result, and each of these is equally valuable when it comes to grading your work.

%What is less valuable is drawing the conclusion that you have solved all the problems with only weak justification.

\section{Drawing a roadmap}

%TODO: 

\subsection{Language support conundrum}

Implemented solution is mainly Java driven

Mining is usually performed in a scripting or data analytics language e.g. R / Python / Groovy

Potential to either choose from using a polyglot VM or adapting the middleware and API to enable integration from multiple languages over a common protocol e.g. HTTP / WS etc

\subsection{Response model is very inflexible}

Improved, more flexible models for response

Impossible to have determined what was needed without doing first.

Improved presentation layers based on any new information added to these models.

\subsection{Truly IDE Agnostic Plugins}

Dataflow is currently uni-directional

However, out of the patterns has emerged the potential of a bidirectional architecture which could serve as the platform for universal, IDE agnostic plugins.

\subsection{Enabling of greater IDE metrics}

With a universal integration point for plugins, we would now have the ability to understand how developers use extensions, providing rich data for research / industry purposes.

