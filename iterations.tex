\chapter{Iterations}

\section{Establishing the Architectural Plumbing}

There are two choices when it comes to creating modular applications, either a plugin based system, such as Java’s OSGi standard, or independent applications that communicate over some form of local or remote connection.

Naturally, this poses questions as to what the nature of scale of the problem would be; in our case, tying multiple information sources together, typical of a medium size corporation do we lean to the solution that better applies to a larger scale.

Extensibility in our case is important; there is an inexorable need for addition of new sources. Plugins are the inherent approach to providing modular extension to a monolithic architecture (Figure \ref{fig:monolithic-approach}). Whereas, as a superset of Service Oriented Architecture, Micro-services are more strongly decoupled (Figure \ref{fig:distributed-approach}), relying on middleware to communicate based on some form of predefined contract.

\begin{figure}[h!]
	\centering
	\includegraphics[width=1\linewidth]{monolithic-approach.png}
	\caption{Component Diagram Illustrating Monolithic Approach}
	\label{fig:monolithic-approach}
\end{figure}

\begin{figure}[h!]
	\centering
	\includegraphics[width=1\linewidth]{distributed-application.png}
	\caption{Component Diagram Illustrating Distributed Approach}
	\label{fig:distributed-approach}
\end{figure}

With large amounts of data exchanged - as is the potential in the envisioned system - a single monolithic application acting as the single processor of the data can rapidly become the chokepoint, subject to temporary mediation via vertical scaling. In contrast, these problems are easily mitigated in distributed architecture through horizontal scaling.

Microservices are no silver bullet approach to proverbial modularity and, or business scalability. In fact, they are often associated with greater development overhead, especially, in the aspect of integration and testing due to lack of immediate feedback on changes; the appropriation of the correct culture, practises and tooling can mitigate many of these overheads. Failing these appropriations, other common anti patterns may emerge, such as distributed monoliths. We will discuss these later.

There are many integration patterns for distributed systems. EIP - a referential principle-driven manual - notes their shortcomings and benefits, but strongly recommends messaging integration patterns as the best approach for most Enterprise problems favouring them based on their high frequency, asynchronous capability.

Implementing a fully functional, distributed messaging solution is complex endeavour; reinventing the wheel is futile when there are ample choices to choose from; RabbitMQ, Apache Kafka, to name a few. Not withstanding the reduction of unnecessary development effort, from the elimination of implementing middleware, duct tape code is still necessary to handle the endpoints connecting to these.

The JVM and its associated languages, are ubiquitous in back office systems in many businesses across the globe; this maturity yields stage to many out-of-the-box solutions to common enterprise problems such as that which we have discussed. Spring is an example of one of the most widely adopted enterprise Java frameworks facilitating these aids.

Spring itself does not however dictate the style of distributed programming but offers a strongly opinionated view on the approaches available. We chose the Command Query Responsibility Segregation (CQRS) approach based on its focus on separation of updating a set of projections - query models - and updating of a central decoupled model - command models. Implementing this pattern is infamously difficult, but with help of an extension based on Spring - Axon Framework - can we reduce much of this effort.  

With the choice of Axon come some beneficial consequences. The framework provides a strongly opinionated view of what is best for the developer; its strong point lays in its approach to Messaging. It defines its own Message Bus implementation - which can be substituted from a set of other choices if needed - as well as handling all of the boiler-plating that would otherwise be needed to glue message endpoint consumption together.

\begin{figure}[h!]
	\centering
	\includegraphics[width=1\linewidth]{distributed-approach-with-axon.png}
	\caption{Component Diagram Illustrating Axon Approach}
	\label{fig:distributed-approach}
\end{figure}

\section{Establishing a Retrieval Mechanism}


Having established, through thorough evaluation, a sound architecture, we proceed to establishing our approach to how data is expected to flow between the defined components. Having defined, and implemented a rough architecture above, we resort to implementing a minimum viable product for retrieving from a client for which we choose a HTTP Web Approach with a single test parameter in the pursuit of a creation of query handler that will return a basic hello world message returning the test parameter appended. Following this approach, we are able to create a very simple test case to verify the functionality (Figure \ref{fig:e2eMockTest}).

\begin{figure}[h!]
		\centering
		\lstinputlisting[language=Python, showspaces=false,                
		showstringspaces=false, tabsize=2]{e2e-acceptance-test-1.py}
		\caption{End To End Test for Mock Example}
		\label{fig:e2eMockTest}
\end{figure}

To achieve the functionality, we define a query object capable of transmitting the test payload. The query object also serves as the contract for those wishing the act on it based on its object signature. We model this process in Figure. Clients receive a query through an interface which they dispatch with context to the message bus and which is then returned from the appropriate source. 

\begin{figure}[h!]
	\centering
	\includegraphics[width=1\linewidth]{first_e2e_test_class_uml.png}
	\caption{UML Class Diagram of the models involved in building and responding to a query}
	\label{fig:firste2etestclassuml}
\end{figure}

\begin{figure}[h!]
	\centering
	\includegraphics[width=0.7\linewidth]{process-diagram.png}
	\caption{Process diagram illustrating flow of query from its handling to response}
	\label{fig:processSequenceQueryFlow}
\end{figure}

By implementing this approach, we end up with a core code example of what will form the implementation pattern throughout this project (Figure \ref{fig:e2eMockClass}).

% TODO : link back to objective about simplicity of this approach.

\begin{figure}[h!]
	\centering
	\lstinputlisting[language=Java, showspaces=false,                
	showstringspaces=false, tabsize=2, breaklines=true]{mock-e2e.java}
	\caption{End To End Implementation for Mock Example}
	\label{fig:e2eMockClass}
\end{figure}

\section{Sprint 1}

\subsection{Sprint Planning}

\subsection{Sprint Implementation}

\subsection{Sprint Review}

\subsection{Sprint Retrospective}


\section{Sprint 2}

\subsection{Sprint Planning}

\subsubsection{Impact Map}

\subsubsection{Selected User Stories}

\begin{enumerate}
	\item As a library / API consumer, to help my consumption, when I enter a command to query for examples, show me how others use it.
	
	\item As a library / API developer, to guide my development and increase awareness, when I enter a command to query for usages, show me how others use it.
	
\end{enumerate}

\subsubsection{Sprint Goal}

Have greater context of API usage implemented through a pull model using mock data.

\subsection{Sprint Implementation}

Moving away from the direct language server specification.

We're no longer interested in the direct support it provides.

We can however learn from its implementation, adapting / reusing the pattern it leverages to address the m x n IDE-language problem.

All previously examined IDEs provide some platform to . (see tab 2.1 background).

Query results are naturally no longer expected to displayed in the hover pop up within the IDE editor windows, so we must take another approach. Instead we consider the ability to render queries as documents; we'll take little care as to the formatting as we focus on the MVP.

%TODO;table of which IDEs support rendering some form of webview.

\subsection{Sprint Review}

More natural feeling, suited more to browsing and discoverability as opposed to being ``fed'' the information.

Proves the feasibility and suitable direction for the project.

\subsection{Sprint Retrospective}

\subsubsection{What went well?}

\subsubsection{What didn't go so well?}

\subsubsection{What do we want to improve?}
\section{Sprint 3}

\subsection{Sprint Planning}

\subsubsection{Impact Map}

\begin{figure}[!h]
	\digraph[width=\textwidth]{workItemsImpactMap}{rankdir=LR; 
		direction=LR;
		
		goal->consumer;
		goal->developer;
		goal->miner;
		
		developer->awareness_of_consumer_related_work;
		developer->contribute_work_items_associated_with_changes;
		
		awareness_of_consumer_related_work->pull_query_of_work_items_for_developer;
		contribute_work_items_associated_with_changes->see_mined_content;
		
		consumer->awareness_of_developer_changes_incidents;
		awareness_of_developer_changes_incidents->pull_query_of_work_items_for_consumer;
		
		
		consumer->contribute_work_items_associated_with_usages_of_api;
		
		contribute_work_items_associated_with_usages_of_api->see_mined_content;
		
		miner->mine_provided_work_items;
		
		mine_provided_work_items->ingest_work_items_from_gitlab;
		mine_provided_work_items->ingest_work_items_from_github;
		
		goal[
			label=<Increase global <br/> awareness of work items>
		];
		
		awareness_of_consumer_related_work[
			label=<Awareness of <br/> consumer related work>
		];
		
		contribute_work_items_associated_with_changes[
			label=<Contribute work items <br/> associated with changes>
		];
		
		mine_provided_work_items[
			label=<Mine provided <br/> work items>
		];
		
		awareness_of_developer_changes_incidents[
			label=<Awareness of developer <br/> changes, incidents ...>
		]
		
		contribute_work_items_associated_with_usages_of_api[
			label=<Contribute work items <br/> associated with <br/> usages of API>
		];
		
		pull_query_of_work_items_for_consumer[
			label=<Pull query for <br/> work items>
		];
		
		see_mined_content[
			label=<See Mined Content>
		];
		
		pull_query_of_work_items_for_developer[
			label=<Pull query for work items>
		]; 
		
		ingest_work_items_from_gitlab[
			label=<Ingest work <br/> items from gitlab>;
		];
		
		ingest_work_items_from_github[
			label=<Ingest work <br/> items from github>;
		];
	}
	\label{fig:workItemsImpactMap}
	\caption{Impact Mapping a route to increased awareness of work items}
\end{figure}

\subsubsection{Selected User Stories}

\subsubsection{Sprint Goal}

Implement a working retrieval and ingesting of work items from GitHub based on a pull query command.

\subsection{Sprint Implementation}

Creation of a shared model of a work item.

\subsection{Sprint Review}

\subsection{Sprint Retrospective}

\subsubsection{What went well?}

\subsubsection{What didn't go so well?}

\subsubsection{What do we want to improve?}
\section{Sprint 4}

\subsection{Sprint Planning}

\subsubsection{Impact Map}

See \ref{fig:workItemsImpactMap}. 

\subsubsection{Selected User Stories}

\subsubsection{Sprint Goal}

Implement a working retrieval and ingesting of work items from GitLab based on a pull query command.

\subsection{Sprint Implementation}

\subsection{Sprint Review}

\subsection{Sprint Retrospective}

\subsubsection{What went well?}

\subsubsection{What didn't go so well?}

\subsubsection{What do we want to improve?}
\section{Sprint 5}

\subsection{Sprint Planning}

\subsubsection{Impact Map}

\begin{figure}[h!]
	\digraph[width=\textwidth]{expertsImpactMap}{rankdir=LR;
		direction=LR; 
		goal->consumer;
		goal->developer;
		goal->miner;
	
		% developer branch
		developer->awareness_of_top_consuming_experts;
		developer->contribute_instances_of_expertise;
		
		contribute_instances_of_expertise->see_sources_mined;
		
		awareness_of_top_consuming_experts->pull_query_cmd_for_experts;
		
		%consumer branch
		
		consumer->awareness_of_domain_experts;
		consumer->contribute_instances_of_consumption_expertise;
		
		awareness_of_domain_experts->pull_query_cmd_for_domain_experts;
		
		contribute_instances_of_consumption_expertise->see_sources_mined;
		
		% miner branch
		miner->mine_provided_expert_data;
		mine_provided_expert_data->ingest_experts_from_gitlab;
		ingest_experts_from_gitlab->miner_gitlab_project_members;
		ingest_experts_from_gitlab->miner_gitlab_commit_authors;
		
		mine_provided_expert_data->ingest_experts_from_github;
		ingest_experts_from_github->miner_github_code_owners_file;
		ingest_experts_from_github->miner_github_collaborators;
		ingest_experts_from_github->miner_github_commit_authors;
			
		
		% metadata supplements
		goal[
			label=< Increase awareness <br/> of experts in field >
		];
		
		%impacts
		
		awareness_of_domain_experts[
			label=<Increased awareness <br/> of domain experts >
		];
		
		contribute_instances_of_consumption_expertise[
			label=<Contribute instances <br/> of consumption expertise>
		];
		
		awareness_of_top_consuming_experts[
			label=< Awareness of top <br/> consuming experts >
		];
		
		contribute_instances_of_expertise[
			label=<Contribute instances <br/> of expertise>
		];
		
		mine_provided_expert_data[
			label=<Mine provided <br/> expert data>
		];
		
		% deliverables
		
		pull_query_cmd_for_domain_experts[
			label=<Pull query command <br/> for domain experts>
		]
		
		see_sources_mined[
			label=<*See sources mined>
		]; 
		
		pull_query_cmd_for_experts[
			label=<Pull query command <br/> for experts>
		];
		
		ingest_experts_from_gitlab[
			label=<Ingest experts from <br/> Gitlab>
		];
		
		ingest_experts_from_github[
			label=<Ingest experts from <br/> Github>
		];
		
		miner_gitlab_project_members[
			label=<Project Members API>;
		];
		
		miner_gitlab_commit_authors[
			label=<Git Commit Authors>;
		];
		
		miner_github_code_owners_file[
			label=<Code <br/> Owners File>;
		];
		
		miner_github_collaborators[
			label=<Collaborators API>;
		];
		
		miner_github_commit_authors[
			label=<Git Commit Authors>
		];
		
	}
	\label{fig:expertsImpactMap}
	\caption{Impact Mapping a route to increased awareness of domain experts}
\end{figure}

\subsubsection{Selected User Stories}

\begin{enumerate}
	\item As a API / library consumer, allow me to query for other experts, so that I am aware of those who may help me.
	
	\item As a API / library developer, allow me to query for experts associated with its consumption, so I can focus more on consumer-driven development.
	
	\item As a repository miner, allow me to provide expert data based on the existing GitHub APIs, to increase contextual information on domain experts.
	
\end{enumerate}

\subsubsection{Sprint Goal}

Implement a working retrieval of experts from GitHub based on a pull query command.

\subsection{Sprint Implementation}

\subsection{Sprint Review}

\subsection{Sprint Retrospective}

\subsubsection{What went well?}

\subsubsection{What didn't go so well?}

\subsubsection{What do we want to improve?}
\section{Sprint 6}

\subsection{Sprint Planning}

\subsubsection{Impact Map}
	
See Figure \ref{fig:expertsImpactMap}.

\subsubsection{Selected User Stories}

\begin{enumerate}
	\item As a repository miner, allow me to provide expert data based on the existing GitLab APIs, to increase contextual information on domain experts.
	
\end{enumerate}

\subsubsection{Sprint Goal}

Implement a working retrieval of experts from GitLab based on a pull query command.

\subsection{Sprint Implementation}

\subsection{Sprint Review}

\subsection{Sprint Retrospective}

\subsubsection{What went well?}

\subsubsection{What didn't go so well?}

\subsubsection{What do we want to improve?}
\section{Sprint 7}

\subsection{Sprint Planning}

\subsubsection{Impact Map}

\begin{center}
	
	\digraph[width=\textwidth]{ds}{rankdir=LR; 
		direction=LR;
		goal->mined_content_provider;
		goal->context_requestors;
		mined_content_provider->prefetch_queries;
		prefetch_queries->cache_query_results;
		context_requestors->provide_context_earlier;
		provide_context_earlier->periodically_broadcast_context;
		
		goal[
			label=<Reduce time taken to <br/> display results <br/> to sub-1 second times>;
		];
		
		mined_content_provider[
			label=<Mined content <br/> provider>
		];
		
		context_requestors[
			label=<Context Requesters>
		];
		
		prefetch_queries[
			label=<Prefetch queries>
		];
		
		cache_query_results[
			label=<Cache query <br/>results>
		];
		
		provide_context_earlier[
			label=<Provide context earlier>
		];
		
		periodically_broadcast_context[
			label=<Periodically broadcast <br/> context>
		];
		
		

	}
	
\end{center}

\subsubsection{Selected User Stories}

\begin{enumerate}
	\item As a content requester, I will periodically broadcast the context to allow others to prepare for any of my queries.
	
	\item As a mined content provider to respond more quickly to queries, I will prefetch and, or cache existing queries.
\end{enumerate}

\subsubsection{Sprint Goal}

Queries currently take a noticeable amount of time to resolve, we must improve the time it takes to return queries.

\subsection{Sprint Implementation}

\subsection{Sprint Review}

\subsection{Sprint Retrospective}

\subsubsection{What went well?}

\subsubsection{What didn't go so well?}

\subsubsection{What do we want to improve?}
\section{Sprint 8}

\subsection{Sprint Planning}

\subsubsection{Impact Map}

\begin{figure}[h!]
	
\end{figure}

\subsubsection{Selected User Stories}

\subsubsection{Sprint Goal}

Improving performance

\subsection{Sprint Implementation}

\subsection{Sprint Review}

\subsection{Sprint Retrospective}

\subsubsection{What went well?}

\subsubsection{What didn't go so well?}

\subsubsection{What do we want to improve?}